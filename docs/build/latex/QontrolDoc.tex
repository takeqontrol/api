%% Generated by Sphinx.
\def\sphinxdocclass{report}
\documentclass[letterpaper,10pt,english]{sphinxmanual}
\ifdefined\pdfpxdimen
   \let\sphinxpxdimen\pdfpxdimen\else\newdimen\sphinxpxdimen
\fi \sphinxpxdimen=.75bp\relax

\PassOptionsToPackage{warn}{textcomp}
\usepackage[utf8]{inputenc}
\ifdefined\DeclareUnicodeCharacter
 \ifdefined\DeclareUnicodeCharacterAsOptional
  \DeclareUnicodeCharacter{"00A0}{\nobreakspace}
  \DeclareUnicodeCharacter{"2500}{\sphinxunichar{2500}}
  \DeclareUnicodeCharacter{"2502}{\sphinxunichar{2502}}
  \DeclareUnicodeCharacter{"2514}{\sphinxunichar{2514}}
  \DeclareUnicodeCharacter{"251C}{\sphinxunichar{251C}}
  \DeclareUnicodeCharacter{"2572}{\textbackslash}
 \else
  \DeclareUnicodeCharacter{00A0}{\nobreakspace}
  \DeclareUnicodeCharacter{2500}{\sphinxunichar{2500}}
  \DeclareUnicodeCharacter{2502}{\sphinxunichar{2502}}
  \DeclareUnicodeCharacter{2514}{\sphinxunichar{2514}}
  \DeclareUnicodeCharacter{251C}{\sphinxunichar{251C}}
  \DeclareUnicodeCharacter{2572}{\textbackslash}
 \fi
\fi
\usepackage{cmap}
\usepackage[T1]{fontenc}
\usepackage{amsmath,amssymb,amstext}
\usepackage{babel}
\usepackage{times}
\usepackage[Bjarne]{fncychap}
\usepackage{sphinx}

\usepackage{geometry}

% Include hyperref last.
\usepackage{hyperref}
% Fix anchor placement for figures with captions.
\usepackage{hypcap}% it must be loaded after hyperref.
% Set up styles of URL: it should be placed after hyperref.
\urlstyle{same}
\addto\captionsenglish{\renewcommand{\contentsname}{Contents:}}

\addto\captionsenglish{\renewcommand{\figurename}{Fig.}}
\addto\captionsenglish{\renewcommand{\tablename}{Table}}
\addto\captionsenglish{\renewcommand{\literalblockname}{Listing}}

\addto\captionsenglish{\renewcommand{\literalblockcontinuedname}{continued from previous page}}
\addto\captionsenglish{\renewcommand{\literalblockcontinuesname}{continues on next page}}

\addto\extrasenglish{\def\pageautorefname{page}}

\setcounter{tocdepth}{1}



\title{QontrolDoc Documentation}
\date{Jun 28, 2020}
\release{0}
\author{Qontrol}
\newcommand{\sphinxlogo}{\vbox{}}
\renewcommand{\releasename}{Release}
\makeindex

\begin{document}

\maketitle
\sphinxtableofcontents
\phantomsection\label{\detokenize{index::doc}}
Welcome to Qontrol’s documentation!

This documentation describes teh functionality of the main commands for the python \sphinxstylestrong{Qontrol} code.
See ‘\sphinxurl{https://qontrol.co.uk/}’


\bigskip\hrule\bigskip




Python Library for interfacing with \sphinxstylestrong{Qontrol} integrated optics control hardware. See {[}our website{]}(\sphinxurl{https://qontrol.co.uk/}) for more details!

This library lets you control your Qontrol controller from Python, via a serial interface.


\chapter{Table of contents}
\label{\detokenize{includeme:qontrol-api}}\label{\detokenize{includeme::doc}}\label{\detokenize{includeme:table-of-contents}}\begin{itemize}
\item {} 
Installation

\item {} 
Usage

\item {} 
Contributing

\item {} 
Credits

\item {} 
License

\end{itemize}


\chapter{Installation}
\label{\detokenize{includeme:installation}}
With python\textgreater{}3.6 installed at the command prompt type:

\sphinxtitleref{pip install qontrol}

The following libraries will also be installed:
\begin{itemize}
\item {} 
serial

\item {} 
collections

\end{itemize}


\chapter{Usage}
\label{\detokenize{includeme:usage}}
See help and documentation at \sphinxurl{https://takeqontrol.github.io/qontrol\_api/}


\chapter{Contributing}
\label{\detokenize{includeme:contributing}}
If you want to contribute to this library contact \sphinxhref{mailto:hello@qontrol.co.uk}{hello@qontrol.co.uk}


\chapter{Credits}
\label{\detokenize{includeme:credits}}
List of contributors and authors:
Qontrol
Joshua W. Silverstone
Raffaele Santagati


\chapter{License}
\label{\detokenize{includeme:license}}
Copyright  \&copy; 2020 Qontrol Systems LLP 2020

Permission is hereby granted, free of charge, to any person obtaining a copy of this software and associated documentation files (the “Software”), to deal in the Software without restriction, including without limitation the rights to use, copy, modify, merge, publish, distribute, sublicense, and/or sell copies of the Software, and to permit persons to whom the Software is furnished to do so, subject to the following conditions:

The above copyright notice and this permission notice shall be included in all copies or substantial portions of the Software.

THE SOFTWARE IS PROVIDED “AS IS”, WITHOUT WARRANTY OF ANY KIND, EXPRESS OR IMPLIED, INCLUDING BUT NOT LIMITED TO THE WARRANTIES OF MERCHANTABILITY, FITNESS FOR A PARTICULAR PURPOSE AND NONINFRINGEMENT. IN NO EVENT SHALL THE AUTHORS OR COPYRIGHT HOLDERS BE LIABLE FOR ANY CLAIM, DAMAGES OR OTHER LIABILITY, WHETHER IN AN ACTION OF CONTRACT, TORT OR OTHERWISE, ARISING FROM, OUT OF OR IN CONNECTION WITH THE SOFTWARE OR THE USE OR OTHER DEALINGS IN THE SOFTWARE.


\chapter{Introduction}
\label{\detokenize{intro:introduction}}\label{\detokenize{intro::doc}}\label{\detokenize{intro:intro}}
qontrol.py is a library for configuring and controlling through python the qontrol products.

This guide will guide you through the process of setting up your Qontrol Q8 device and give you a description of the basic commands.


\chapter{Getting Started Guide}
\label{\detokenize{guide/getting_started::doc}}\label{\detokenize{guide/getting_started:getting-started-guide}}
This Getting started document will guide you through the first configuration and test of the q8 unit
attached to a Qontrol motherboard.

The operations described in this guide require:
\begin{itemize}
\item {} 
a Qontrol controller (e.g. Q8iv or Q8)

\item {} 
a Qontrol motherboard (e.g. BP8)

\item {} 
the corresponding power supply (e.g. PS15KIT)

\item {} 
a computer connected to the internet to let the system download the serial port USB drivers

\item {} 
(optionally) a program for serial port communication (e.g. Teraterm for windows, CoolTerm for Mac Osx, Linux)

\end{itemize}


\section{Connection of the unit}
\label{\detokenize{guide/getting_started:connection-of-the-unit}}
This procedure is valid for both the Q8 and the Q8iv products.
\begin{itemize}
\item {} 
Insert the Q8iv (or any other compatible control unit) in one of the backplanes/motherboards (e.g. BP8):

\sphinxincludegraphics[width=75\sphinxpxdimen]{{Q8iv_stil}.jpg} \sphinxincludegraphics[width=100\sphinxpxdimen]{{BP8-top}.jpg}

\item {} 
In the Q8iV the 5 leds indicate (from the bottom): pwr, rx, tx, activity, err. In the Q8b the 3 leds indicate: pwr, activity, err

\item {} 
Connect the backplane (motherboard) unit to the computer using a cable/adapter with a USB mini b female plug at one of the two ends, like the one shown below:

\noindent\sphinxincludegraphics[width=100\sphinxpxdimen]{{usbminib}.jpg}

\item {} 
Power the unit, using a compatible power supply (e.g. PS15KIT):

\noindent\sphinxincludegraphics[width=100\sphinxpxdimen]{{PS15KIT}.jpg}

\item {} 
Al the side LEDs in the units should progressively turn on and off again leaving only the bottom green LEDs on, while the units are in idle.

\end{itemize}

Whole system:

\sphinxincludegraphics[width=500\sphinxpxdimen]{{Qontrol-system-overview}.png}


\section{Configuration of the serial communication}
\label{\detokenize{guide/getting_started:configuration-of-the-serial-communication}}

\subsection{Controlling the unit using the provided Qontrol API serial communication commands}
\label{\detokenize{guide/getting_started:controlling-the-unit-using-the-provided-qontrol-api-serial-communication-commands}}
This API provides also a command line interface for direct communication with the Qontrol unit.
The program is called run\_interactive\_shell(). To run it follow these steps: start python, import qontrol and run the interactive shell.
\begin{description}
\item[{::}] \leavevmode
\$ python

\fvset{hllines={, ,}}%
\begin{sphinxVerbatim}[commandchars=\\\{\}]
\PYG{g+gp}{\PYGZgt{}\PYGZgt{}\PYGZgt{} }\PYG{k+kn}{import} \PYG{n+nn}{qontrol}
\end{sphinxVerbatim}

\fvset{hllines={, ,}}%
\begin{sphinxVerbatim}[commandchars=\\\{\}]
\PYG{g+gp}{\PYGZgt{}\PYGZgt{}\PYGZgt{} }\PYG{n}{qontrol}\PYG{o}{.}\PYG{n}{run\PYGZus{}interactive\PYGZus{}shell}\PYG{p}{(}\PYG{p}{)}
\end{sphinxVerbatim}

\end{description}

Select your controller from the list of available serial devices and you are ready to go.


\subsection{Controlling the unit using a serial communication software}
\label{\detokenize{guide/getting_started:controlling-the-unit-using-a-serial-communication-software}}\begin{quote}

In addition, serial communication software in any operating system (OS) can be used to control the units, some examples:
\begin{itemize}
\item {} 
Teraterm (Windows)

\item {} 
CoolTerm (Mac)

\item {} 
Terminal/Command line (Linux)

\end{itemize}
\end{quote}

\sphinxstylestrong{General Configuration settings.}

\sphinxstyleemphasis{Serial parameters settings} :


\begin{savenotes}\sphinxattablestart
\centering
\begin{tabulary}{\linewidth}[t]{|T|T|}
\hline
\sphinxstyletheadfamily 
Setting
&\sphinxstyletheadfamily 
Value
\\
\hline
Data bits
&
8
\\
\hline
Stop bit
&
1
\\
\hline
Par. check
&
None
\\
\hline
Flow ctrl.
&
None
\\
\hline
Baud rate
&
115200
\\
\hline
\end{tabulary}
\par
\sphinxattableend\end{savenotes}

\sphinxstylestrong{Windows Teraterm Configuration}

\sphinxstylestrong{Mac OSX CoolTerm Configuration}
\begin{itemize}
\item {} 
Open a Terminal

\item {} 
check the name of the device with the command:

ls /dev/tty.usb*

\item {} 
Example of output:
/dev/tty.usbserial-FT31EUVZ

\item {} 
the name “FT31EUVZ” identifies the connection to the Qontrol motherboard

\item {} 
Open CoolTerm and select options

\end{itemize}

\noindent\sphinxincludegraphics[width=350\sphinxpxdimen]{{CoolTerm0}.png}
\begin{itemize}
\item {} 
Select the correct device and the proper settings

\item {} 
Open CoolTerm and select the appropriate options

\end{itemize}

\noindent\sphinxincludegraphics[width=350\sphinxpxdimen]{{CoolTerm1}.png}
\begin{itemize}
\item {} 
Select Ok and start typing the commands

\end{itemize}

\noindent\sphinxincludegraphics[width=350\sphinxpxdimen]{{CoolTerm2}.png}

\sphinxstylestrong{Linux Command Line}

In Linux is also possible to use terminal software such as \sphinxstylestrong{minicorn}
\begin{itemize}
\item {} 
Check the name of the device

ls /dev/tty.usb*

\item {} 
Serial ports devices will appear as /dev/ttyS\#

\item {} 
To change the serial port configuration use the command \sphinxstylestrong{‘ssty’}, use the command
“man stty”
for specific operation details

\item {} 
Example to set the Baudrate to 115200 and odd parity
stty -F /dev/ttyS\# 115200 parodd

\item {} 
Issue Comands using the \sphinxstylestrong{“echo”} command
echo ‘vipall?’ \textgreater{} /dev/ttyusb\#

\item {} 
Read the data with \sphinxstylestrong{cat}:
cat /dev/ttyusb\#

\end{itemize}


\section{Main commands and error codes}
\label{\detokenize{guide/getting_started:main-commands-and-error-codes}}
From the serial interface you can always obtain te list of commands by typing:
\begin{description}
\item[{::}] \leavevmode
\textgreater{} help

\end{description}

\sphinxstylestrong{CORE COMMAND SET}
\begin{itemize}
\item {} 
Set voltage of a specific “port” to “value” (V) -\textgreater{} v{[}port{]}={[}value{]}

\item {} 
Set current of a specific “port” to “value” (mA) -\textgreater{} i{[}port{]}={[}value{]}

\item {} 
Read voltage of a specific “port” (V) - \textgreater{} v{[}port{]}?

\item {} 
Read current of a specific “port” (mA) - \textgreater{} i{[}port{]}?

\item {} 
Read voltage current and power on all the ports (V, mA, mW)-\textgreater{} vipall?

\item {} 
Set max voltage limit (V) to a “value” of a specific “port” -\textgreater{} vmax{[}port{]}={[}value{]}

\item {} 
Set max current limit (mA) to a “value” of a specific “port” -\textgreater{} imax{[}port{]}={[}value{]}

\end{itemize}


\begin{savenotes}\sphinxattablestart
\centering
\begin{tabulary}{\linewidth}[t]{|T|T|}
\hline
\sphinxstyletheadfamily 
Error Code
&\sphinxstyletheadfamily 
Description
\\
\hline
00
&
Unknown error
\\
\hline
01
&
Overvoltage
\\
\hline
02
&
Overcurrent
\\
\hline
03
&
Power error
\\
\hline
04
&
Calibration error
\\
\hline
10
&
Unrecognised command
\\
\hline
11
&
Unrecognised parameter
\\
\hline
12
&
Unrecognised port
\\
\hline
13
&
Operation forbidden
\\
\hline
14:00
&
Instruction buffer overflow
\\
\hline
14:01
&
Single instruction overflow
\\
\hline
15:X0
&
Serial overflow detected
\\
\hline
15:X1
&
Serial framing error detected
\\
\hline
16
&
Internal software error
\\
\hline
\end{tabulary}
\par
\sphinxattableend\end{savenotes}


\section{Frequently asked questions (FAQ) and basic troubleshooting}
\label{\detokenize{guide/getting_started:frequently-asked-questions-faq-and-basic-troubleshooting}}
** Which operating systems are supported? **

Our Python API can be used on any modern OS. All of them that can control a serial port can also use our products in any other language, if you are willing to program this yourself.

** Why does the backplane error light turn on when drivers are activated?**

The LSD is inserted the wrong way around!


\section{Running the example code}
\label{\detokenize{guide/getting_started:running-the-example-code}}
The example code \sphinxstyleemphasis{example.py} can be found in the intallation directory of \sphinxstyleemphasis{qontrol.py}.

To run it sympy type:

\$ python example.py

The example will set some voltages on your device so it is important to check that no sensitive equipment is connected when running it.

If the example code runs successfully, the lights on your qontroller will flash as commands are transmitted, and output channels are energised.
The qontroller’s device ID will be read out, as well as its number of channels. The voltage on each channel will be briefly set to the channel’s number,
then return to zero (e.g. channel 3 is set to 3 V). The current of each channel will be read out. Next, we’ll take a look at exactly what’s inside.


\section{API basics}
\label{\detokenize{guide/getting_started:api-basics}}
First, we read in the qontrol.py API with an import call.

\fvset{hllines={, ,}}%
\begin{sphinxVerbatim}[commandchars=\\\{\}]
\PYG{k+kn}{import} \PYG{n+nn}{qontrol}
\end{sphinxVerbatim}

To initialise an output device (or daisy-chain), like a Q8iv, first use a definition like

\fvset{hllines={, ,}}%
\begin{sphinxVerbatim}[commandchars=\\\{\}]
\PYG{n}{q} \PYG{o}{=} \PYG{n}{qontrol}\PYG{o}{.}\PYG{n}{QXOutput}\PYG{p}{(}\PYG{n}{serial\PYGZus{}port\PYGZus{}name} \PYG{o}{=} \PYG{l+s+s2}{\PYGZdq{}}\PYG{l+s+s2}{COM1}\PYG{l+s+s2}{\PYGZdq{}}\PYG{p}{)}
\end{sphinxVerbatim}

where q is our qontroller object, which stores information about the hardware, such as its device ID (q.device\_id) and number of channels (q.n\_chs), and handles all communications and commands.
COM1 is an example of what the connected serial port name might be on Windows.

Port numbers greater than 10 must be written like

\fvset{hllines={, ,}}%
\begin{sphinxVerbatim}[commandchars=\\\{\}]
\PYG{o}{/}\PYG{o}{/}\PYG{o}{.}\PYG{o}{/}\PYG{n}{COM42}\PYG{o}{.}
\end{sphinxVerbatim}

On Mac or Linux, this will look something like

\fvset{hllines={, ,}}%
\begin{sphinxVerbatim}[commandchars=\\\{\}]
\PYG{o}{/}\PYG{n}{dev}\PYG{o}{/}\PYG{n}{tty}\PYG{o}{.}\PYG{n}{usbserial}\PYG{o}{\PYGZhy{}}\PYG{n}{FT123ABC}\PYG{o}{.}
\end{sphinxVerbatim}

When your code is done with the hardware, it’s good practice to close the connection with a call to

\fvset{hllines={, ,}}%
\begin{sphinxVerbatim}[commandchars=\\\{\}]
\PYG{n}{q}\PYG{o}{.}\PYG{n}{close}\PYG{p}{(}\PYG{p}{)}
\end{sphinxVerbatim}

Controlling outputs, and reading inputs is easy.

Writing to the qontroller object’s v or i arrays (using standard Python array indexing) sets the output for channel with that index.

Reading from those arrays reads the input values back from the hardware. For example, we can set the voltage on output channel 3 to 4.5V and read back the current (in mA) assigning the value to the variable “measured\_current” with the code:

\fvset{hllines={, ,}}%
\begin{sphinxVerbatim}[commandchars=\\\{\}]
\PYG{n}{q}\PYG{o}{.}\PYG{n}{v}\PYG{p}{[}\PYG{l+m+mi}{3}\PYG{p}{]} \PYG{o}{=} \PYG{l+m+mf}{4.5}
\PYG{n}{measured\PYGZus{}current} \PYG{o}{=} \PYG{n}{q}\PYG{o}{.}\PYG{n}{i}\PYG{p}{[}\PYG{l+m+mi}{3}\PYG{p}{]}
\end{sphinxVerbatim}

We can do bulk changes to channels using slices.

\fvset{hllines={, ,}}%
\begin{sphinxVerbatim}[commandchars=\\\{\}]
\PYG{n}{q}\PYG{o}{.}\PYG{n}{v}\PYG{p}{[}\PYG{l+m+mi}{2}\PYG{p}{:}\PYG{l+m+mi}{5}\PYG{p}{]} \PYG{o}{=} \PYG{l+m+mf}{4.5}
\PYG{n}{measured\PYGZus{}current} \PYG{o}{=} \PYG{n}{q}\PYG{o}{.}\PYG{n}{i}\PYG{p}{[}\PYG{l+m+mi}{2}\PYG{p}{:}\PYG{l+m+mi}{5}\PYG{p}{]}
\end{sphinxVerbatim}

The slice character in Python, :, means either “everything between two indices” (e.g. v{[}2:5{]}), “everything from the beginning until an index” (e.g. v{[}:5{]}), or “everything from an index until the end” (e.g. v{[}2:{]}).


\section{Notes and disclaimer}
\label{\detokenize{guide/getting_started:notes-and-disclaimer}}
If you find an error in this document, or have suggestions for how we could make it better, please do get in touch with us at \sphinxhref{mailto:support@qontrol.co.uk}{support@qontrol.co.uk} with your comments.

The information provided in this document is believed to be accurate at the time of publication. It is provided for information only, ‘as is’, and without guarantee of any kind.

Qontrol Systems LLP, its subsidiaries and associates accept no liability for damage to equipment, hardware, or the customer application, or for labour costs incurred due to the information contained in this document.


\chapter{qontrol\_api}
\label{\detokenize{modules:qontrol-api}}\label{\detokenize{modules::doc}}

\section{qontrol module}
\label{\detokenize{qontrol:qontrol-module}}\label{\detokenize{qontrol::doc}}\label{\detokenize{qontrol:module-qontrol}}\index{qontrol (module)}\index{ChannelVector (class in qontrol)}

\begin{fulllineitems}
\phantomsection\label{\detokenize{qontrol:qontrol.ChannelVector}}\pysiglinewithargsret{\sphinxbfcode{\sphinxupquote{class }}\sphinxcode{\sphinxupquote{qontrol.}}\sphinxbfcode{\sphinxupquote{ChannelVector}}}{\emph{base\_list}}{}
Bases: \sphinxcode{\sphinxupquote{object}}

Custom list class which has a fixed length but mutable (typed) elements, and which phones home when its elements are read or modified.
\index{get\_handle (qontrol.ChannelVector attribute)}

\begin{fulllineitems}
\phantomsection\label{\detokenize{qontrol:qontrol.ChannelVector.get_handle}}\pysigline{\sphinxbfcode{\sphinxupquote{get\_handle}}\sphinxbfcode{\sphinxupquote{ = None}}}
\end{fulllineitems}

\index{set\_handle (qontrol.ChannelVector attribute)}

\begin{fulllineitems}
\phantomsection\label{\detokenize{qontrol:qontrol.ChannelVector.set_handle}}\pysigline{\sphinxbfcode{\sphinxupquote{set\_handle}}\sphinxbfcode{\sphinxupquote{ = None}}}
\end{fulllineitems}

\index{valid\_types (qontrol.ChannelVector attribute)}

\begin{fulllineitems}
\phantomsection\label{\detokenize{qontrol:qontrol.ChannelVector.valid_types}}\pysigline{\sphinxbfcode{\sphinxupquote{valid\_types}}\sphinxbfcode{\sphinxupquote{ = (\textless{}class 'int'\textgreater{}, \textless{}class 'float'\textgreater{})}}}
\end{fulllineitems}


\end{fulllineitems}

\index{QXOutput (class in qontrol)}

\begin{fulllineitems}
\phantomsection\label{\detokenize{qontrol:qontrol.QXOutput}}\pysiglinewithargsret{\sphinxbfcode{\sphinxupquote{class }}\sphinxcode{\sphinxupquote{qontrol.}}\sphinxbfcode{\sphinxupquote{QXOutput}}}{\emph{*args}, \emph{**kwargs}}{}
Bases: {\hyperref[\detokenize{qontrol:qontrol.Qontroller}]{\sphinxcrossref{\sphinxcode{\sphinxupquote{qontrol.Qontroller}}}}}
\index{get\_all\_values() (qontrol.QXOutput method)}

\begin{fulllineitems}
\phantomsection\label{\detokenize{qontrol:qontrol.QXOutput.get_all_values}}\pysiglinewithargsret{\sphinxbfcode{\sphinxupquote{get\_all\_values}}}{\emph{para='V'}}{}
\end{fulllineitems}

\index{get\_value() (qontrol.QXOutput method)}

\begin{fulllineitems}
\phantomsection\label{\detokenize{qontrol:qontrol.QXOutput.get_value}}\pysiglinewithargsret{\sphinxbfcode{\sphinxupquote{get\_value}}}{\emph{ch}, \emph{para='V'}}{}
\end{fulllineitems}

\index{set\_all\_values() (qontrol.QXOutput method)}

\begin{fulllineitems}
\phantomsection\label{\detokenize{qontrol:qontrol.QXOutput.set_all_values}}\pysiglinewithargsret{\sphinxbfcode{\sphinxupquote{set\_all\_values}}}{\emph{para='V'}, \emph{values=0}}{}
Convenience function for slicing up set commands into vectors for each module and transmitting.
\begin{quote}

para:          Parameter to set \{‘V’ or ‘I’\}
values:        Either float/int or list of float/int of length n\_chs
\end{quote}

\end{fulllineitems}

\index{set\_value() (qontrol.QXOutput method)}

\begin{fulllineitems}
\phantomsection\label{\detokenize{qontrol:qontrol.QXOutput.set_value}}\pysiglinewithargsret{\sphinxbfcode{\sphinxupquote{set\_value}}}{\emph{ch}, \emph{para='V'}, \emph{new=0}}{}
\end{fulllineitems}


\end{fulllineitems}

\index{Qontroller (class in qontrol)}

\begin{fulllineitems}
\phantomsection\label{\detokenize{qontrol:qontrol.Qontroller}}\pysiglinewithargsret{\sphinxbfcode{\sphinxupquote{class }}\sphinxcode{\sphinxupquote{qontrol.}}\sphinxbfcode{\sphinxupquote{Qontroller}}}{\emph{*args}, \emph{**kwargs}}{}
Bases: \sphinxcode{\sphinxupquote{object}}

Super class which handles serial communication, device identification, and logging.
\begin{quote}

device\_id = None                                        Device ID
serial\_port = None                                      Serial port object
serial\_port\_name = None                         Name of serial port, eg ‘COM1’ or ‘/dev/tty1’
error\_desc\_dict = Q8x\_ERRORS                    Error code descriptions
log = fifo(maxlen = 256)                        Log FIFO of sent commands and received errors
log\_handler = None                                      Function which catches log dictionaries
log\_to\_stdout = True                            Copy new log entries to stdout
response\_timeout = 0.050                        Timeout for response or error to commands
inter\_response\_timeout = 0.020          Timeout for response or error to get commands
\end{quote}

Log handler:
The log handler may be used to catch and dynamically handle certain errors, as they arise. In the following example, it is set up to raise a RuntimeError upon reception of errors E01, E02, and E03:
\begin{quote}

q = Qontroller()

fatal\_errors = {[}1, 2, 3{]}
\begin{description}
\item[{def my\_log\_handler(err\_dict):}] \leavevmode\begin{description}
\item[{if err\_dict{[}‘type’{]} is ‘err’ and err\_dict{[}‘id’{]} in fatal\_errors:}] \leavevmode
raise RuntimeError(‘Caught Qontrol error “\{1\}” at \{0\} ms’.format(1000*err\_dict{[}‘proctime’{]}, err\_dict{[}‘desc’{]}))

\end{description}

\end{description}

q.log\_handler = my\_log\_handler
\end{quote}
\index{close() (qontrol.Qontroller method)}

\begin{fulllineitems}
\phantomsection\label{\detokenize{qontrol:qontrol.Qontroller.close}}\pysiglinewithargsret{\sphinxbfcode{\sphinxupquote{close}}}{}{}
Release resources

\end{fulllineitems}

\index{issue\_binary\_command() (qontrol.Qontroller method)}

\begin{fulllineitems}
\phantomsection\label{\detokenize{qontrol:qontrol.Qontroller.issue_binary_command}}\pysiglinewithargsret{\sphinxbfcode{\sphinxupquote{issue\_binary\_command}}}{\emph{command\_id}, \emph{ch=None}, \emph{BCAST=0}, \emph{ALLCH=0}, \emph{ADDM=0}, \emph{RW=0}, \emph{ACT=0}, \emph{DEXT=0}, \emph{value\_int=0}, \emph{addr\_id\_num=0}, \emph{n\_lines\_requested=2147483648}, \emph{target\_errors=None}, \emph{output\_regex='(.*)'}, \emph{special\_timeout=None}}{}
Transmit command ({[}command\_id{]}{[}ch{]}{[}operator{]}{[}value{]}) to device, collect response.
\begin{quote}

command\_id:             Command descriptor, either int (command index) or str (command name).
ch:                     Channel address (0x0000..0xFFFF for ADDM=0, 0x00..0xFF for ADDM=1).
BCAST,
\begin{quote}

ALLCH,
ADDM,
RW,
ACT,
DEXT:                  Header byte bits. See Programming Manual for full description.
\end{quote}

value\_int:              Data, either int (DEXT=0) or list of int (DEXT=1).
addr\_id\_num:    For device-wise addressing mode (ADDM=1) only, hex device ID code.
\end{quote}

All other arguments same as those for issue\_command()

\end{fulllineitems}

\index{issue\_command() (qontrol.Qontroller method)}

\begin{fulllineitems}
\phantomsection\label{\detokenize{qontrol:qontrol.Qontroller.issue_command}}\pysiglinewithargsret{\sphinxbfcode{\sphinxupquote{issue\_command}}}{\emph{command\_id}, \emph{ch=None}, \emph{operator=''}, \emph{value=None}, \emph{n\_lines\_requested=2147483648}, \emph{target\_errors=None}, \emph{output\_regex='(.*)'}, \emph{special\_timeout=None}}{}
Transmit command ({[}command\_id{]}{[}ch{]}{[}operator{]}{[}value{]}) to device, collect response.
\begin{quote}

command\_id                      Command header (e.g. ‘v’ in ‘v7=1.0’)
ch                                      Channel index to apply command to (e.g. ‘7’ in ‘v7=1.0’)
operator                        Type of command in \{?, =\} (e.g. ‘=’ in ‘v7=1.0’)
value                           Value of set command (e.g. ‘1.0’ in ‘v7=1.0’)
n\_lines\_requested       Lines of data (not error) to stop after receiving, or timeout
target\_errors           Error numbers which will be raised as RuntimeError
special\_timeout         Timeout to use for this command only (!= self.response\_timeout)
\end{quote}

\end{fulllineitems}

\index{log\_append() (qontrol.Qontroller method)}

\begin{fulllineitems}
\phantomsection\label{\detokenize{qontrol:qontrol.Qontroller.log_append}}\pysiglinewithargsret{\sphinxbfcode{\sphinxupquote{log\_append}}}{\emph{type='err'}, \emph{id=''}, \emph{ch=0}, \emph{value=0}, \emph{desc=''}, \emph{raw=''}}{}
Append an event to the log, adding both a calendar- and a process-timestamp.”

\end{fulllineitems}

\index{parse\_error() (qontrol.Qontroller method)}

\begin{fulllineitems}
\phantomsection\label{\detokenize{qontrol:qontrol.Qontroller.parse_error}}\pysiglinewithargsret{\sphinxbfcode{\sphinxupquote{parse\_error}}}{\emph{error\_str}}{}
Parse an encoded error (e.g. E02:07) into its code, channel, and human-readable description.

\end{fulllineitems}

\index{print\_log() (qontrol.Qontroller method)}

\begin{fulllineitems}
\phantomsection\label{\detokenize{qontrol:qontrol.Qontroller.print_log}}\pysiglinewithargsret{\sphinxbfcode{\sphinxupquote{print\_log}}}{\emph{n=None}}{}
Print the n last log entries. If n == None, print all log entries.

\end{fulllineitems}

\index{receive() (qontrol.Qontroller method)}

\begin{fulllineitems}
\phantomsection\label{\detokenize{qontrol:qontrol.Qontroller.receive}}\pysiglinewithargsret{\sphinxbfcode{\sphinxupquote{receive}}}{}{}
Low-level receive data method which also checks for errors.

\end{fulllineitems}

\index{transmit() (qontrol.Qontroller method)}

\begin{fulllineitems}
\phantomsection\label{\detokenize{qontrol:qontrol.Qontroller.transmit}}\pysiglinewithargsret{\sphinxbfcode{\sphinxupquote{transmit}}}{\emph{command\_string}, \emph{binary\_mode=False}}{}
Low-level transmit data method. command\_string can be of type str or bytearray

\end{fulllineitems}

\index{wait() (qontrol.Qontroller method)}

\begin{fulllineitems}
\phantomsection\label{\detokenize{qontrol:qontrol.Qontroller.wait}}\pysiglinewithargsret{\sphinxbfcode{\sphinxupquote{wait}}}{\emph{seconds=0.0}}{}
Do nothing while watching for errors on the serial bus.

\end{fulllineitems}


\end{fulllineitems}

\index{run\_interactive\_shell() (in module qontrol)}

\begin{fulllineitems}
\phantomsection\label{\detokenize{qontrol:qontrol.run_interactive_shell}}\pysiglinewithargsret{\sphinxcode{\sphinxupquote{qontrol.}}\sphinxbfcode{\sphinxupquote{run\_interactive\_shell}}}{}{}
Interactive shell for interacting directly with Qontrol hardware.

\end{fulllineitems}



\chapter{Indices and tables}
\label{\detokenize{index:indices-and-tables}}\begin{itemize}
\item {} 
\DUrole{xref,std,std-ref}{genindex}

\item {} 
\DUrole{xref,std,std-ref}{modindex}

\item {} 
\DUrole{xref,std,std-ref}{search}

\end{itemize}


\renewcommand{\indexname}{Python Module Index}
\begin{sphinxtheindex}
\def\bigletter#1{{\Large\sffamily#1}\nopagebreak\vspace{1mm}}
\bigletter{q}
\item {\sphinxstyleindexentry{qontrol}}\sphinxstyleindexpageref{qontrol:\detokenize{module-qontrol}}
\end{sphinxtheindex}

\renewcommand{\indexname}{Index}
\printindex
\end{document}