%% Generated by Sphinx.
\def\sphinxdocclass{report}
\documentclass[letterpaper,10pt,english]{sphinxmanual}
\ifdefined\pdfpxdimen
   \let\sphinxpxdimen\pdfpxdimen\else\newdimen\sphinxpxdimen
\fi \sphinxpxdimen=.75bp\relax

\PassOptionsToPackage{warn}{textcomp}
\usepackage[utf8]{inputenc}
\ifdefined\DeclareUnicodeCharacter
% support both utf8 and utf8x syntaxes
  \ifdefined\DeclareUnicodeCharacterAsOptional
    \def\sphinxDUC#1{\DeclareUnicodeCharacter{"#1}}
  \else
    \let\sphinxDUC\DeclareUnicodeCharacter
  \fi
  \sphinxDUC{00A0}{\nobreakspace}
  \sphinxDUC{2500}{\sphinxunichar{2500}}
  \sphinxDUC{2502}{\sphinxunichar{2502}}
  \sphinxDUC{2514}{\sphinxunichar{2514}}
  \sphinxDUC{251C}{\sphinxunichar{251C}}
  \sphinxDUC{2572}{\textbackslash}
\fi
\usepackage{cmap}
\usepackage[T1]{fontenc}
\usepackage{amsmath,amssymb,amstext}
\usepackage{babel}



\usepackage{times}
\expandafter\ifx\csname T@LGR\endcsname\relax
\else
% LGR was declared as font encoding
  \substitutefont{LGR}{\rmdefault}{cmr}
  \substitutefont{LGR}{\sfdefault}{cmss}
  \substitutefont{LGR}{\ttdefault}{cmtt}
\fi
\expandafter\ifx\csname T@X2\endcsname\relax
  \expandafter\ifx\csname T@T2A\endcsname\relax
  \else
  % T2A was declared as font encoding
    \substitutefont{T2A}{\rmdefault}{cmr}
    \substitutefont{T2A}{\sfdefault}{cmss}
    \substitutefont{T2A}{\ttdefault}{cmtt}
  \fi
\else
% X2 was declared as font encoding
  \substitutefont{X2}{\rmdefault}{cmr}
  \substitutefont{X2}{\sfdefault}{cmss}
  \substitutefont{X2}{\ttdefault}{cmtt}
\fi


\usepackage[Bjarne]{fncychap}
\usepackage{sphinx}

\fvset{fontsize=\small}
\usepackage{geometry}


% Include hyperref last.
\usepackage{hyperref}
% Fix anchor placement for figures with captions.
\usepackage{hypcap}% it must be loaded after hyperref.
% Set up styles of URL: it should be placed after hyperref.
\urlstyle{same}
\addto\captionsenglish{\renewcommand{\contentsname}{Contents:}}

\usepackage{sphinxmessages}
\setcounter{tocdepth}{1}



\title{QontrolDoc Documentation}
\date{Jun 05, 2020}
\release{0}
\author{Qontrol}
\newcommand{\sphinxlogo}{\vbox{}}
\renewcommand{\releasename}{Release}
\makeindex
\begin{document}

\pagestyle{empty}
\sphinxmaketitle
\pagestyle{plain}
\sphinxtableofcontents
\pagestyle{normal}
\phantomsection\label{\detokenize{index::doc}}
Welcome to Qontrol’s documentation!

This documentation describes teh functionality of the main commands for the python \sphinxstylestrong{Qontrol} code.
See ‘\sphinxurl{https://qontrol.co.uk/}’


\bigskip\hrule\bigskip




qontrol.py is a library for configuring and controlling through python the qontrol products.

This guide will guide you through the process of setting up your Qontrol Q8 device and give you a description of the basic commands.


\chapter{Getting Started Guide}
\label{\detokenize{guide/getting_started:getting-started-guide}}\label{\detokenize{guide/getting_started::doc}}
This Getting started document will guide you through the first configuration and test of the q8 unit
attached to a Qontrol motherboard.

The operations described in this guide require:
\begin{itemize}
\item {} 
a Qontrol controller (e.g. Q8iv or Q8)

\item {} 
a Qontrol motherboard (e.g. BP8)

\item {} 
the corresponding power supply (e.g. PS15KIT)

\item {} 
a computer connected to the internet to let the system download the serial port USB drivers

\item {} 
(optionally) a program for serial port communication (e.g. Teraterm for windows, CoolTerm for Mac Osx, Linux)

\end{itemize}


\section{Connection of the unit}
\label{\detokenize{guide/getting_started:connection-of-the-unit}}
This procedure is valid for both the Q8 and the Q8iv products.
\begin{itemize}
\item {} 
Insert the Q8iv (or any other compatible control unit) in one of the backplanes/motherboards (e.g. BP8):

\sphinxincludegraphics[width=75\sphinxpxdimen]{{Q8iv_stil}.jpg} \sphinxincludegraphics[width=100\sphinxpxdimen]{{BP8-top}.jpg}

\item {} 
Connect the backplane (motherboard) unit to the computer using a cable/adapter with a USB mini b female plug at one of the two ends, like the one shown below:

\noindent\sphinxincludegraphics[width=100\sphinxpxdimen]{{usbminib}.jpg}

\item {} 
Power the unit, using a compatible power supply (e.g. PS15KIT):

\noindent\sphinxincludegraphics[width=100\sphinxpxdimen]{{PS15KIT}.jpg}

\item {} 
Al the side LEDs in the units should progressively turn on and off again leaving only the bottom green LEDs on, while the units are in idle.

\end{itemize}

Whole system:
\sphinxincludegraphics[width=100\sphinxpxdimen]{{Qontrol-system-overview}.png}


\section{Configuration of the serial communication}
\label{\detokenize{guide/getting_started:configuration-of-the-serial-communication}}

\subsection{Controlling the unit using the provided Qontrol API serial communication commands}
\label{\detokenize{guide/getting_started:controlling-the-unit-using-the-provided-qontrol-api-serial-communication-commands}}
This API provides also a command line interface for direct communication with the Qontrol unit.

To set the communication


\subsection{Controlling the unit using a serial communication software}
\label{\detokenize{guide/getting_started:controlling-the-unit-using-a-serial-communication-software}}\begin{quote}

Serial communication software in any operating system (OS) can be used to control the units, some examples:
\begin{itemize}
\item {} 
Teraterm (Windows)

\item {} 
CoolTerm (Mac)

\item {} 
Terminal/Command line (Linux)

\end{itemize}
\end{quote}

\sphinxstylestrong{General Configuration settings.}

\sphinxstyleemphasis{Serial parameters settings} :


\begin{savenotes}\sphinxattablestart
\centering
\begin{tabulary}{\linewidth}[t]{|T|T|}
\hline
\sphinxstyletheadfamily 
Setting
&\sphinxstyletheadfamily 
Value
\\
\hline
Data bits
&
8
\\
\hline
Stop bit
&
1
\\
\hline
Par. check
&
None
\\
\hline
Flow ctrl.
&
None
\\
\hline
Baud rate
&
115200
\\
\hline
\end{tabulary}
\par
\sphinxattableend\end{savenotes}

\sphinxstylestrong{Windows Teraterm Configuration}

\sphinxstylestrong{Mac OSX CoolTerm Configuration}
\begin{itemize}
\item {} 
Open a Terminal

\item {} 
check the name of the device with the command:

ls /dev/tty.usb*

\item {} 
Example of output:
/dev/tty.usbserial\sphinxhyphen{}FT31EUVZ

\item {} 
the name “FT31EUVZ” identifies the connection to the Qontrol motherboard

\item {} 
Open CoolTerm and select options

\end{itemize}

\noindent\sphinxincludegraphics[width=350\sphinxpxdimen]{{CoolTerm0}.png}
\begin{itemize}
\item {} 
Select the correct device and the proper settings

\item {} 
Open CoolTerm and select the appropriate options

\end{itemize}

\noindent\sphinxincludegraphics[width=350\sphinxpxdimen]{{CoolTerm1}.png}
\begin{itemize}
\item {} 
Select Ok and start typing the commands

\end{itemize}

\noindent\sphinxincludegraphics[width=350\sphinxpxdimen]{{CoolTerm2}.png}

\sphinxstylestrong{Linux Command Line}

In Linux is also possible to use terminal software such as \sphinxstylestrong{minicorn}
\begin{itemize}
\item {} 
Check the name of the device

ls /dev/tty.usb*

\item {} 
Serial ports devices will appear as /dev/ttyS\#

\item {} 
To change the serial port configuration use the command \sphinxstylestrong{‘ssty’}, use the command
“man stty”
for specific operation details

\item {} 
Example to set the Baudrate to 115200 and odd parity
stty \sphinxhyphen{}F /dev/ttyS\# 115200 parodd

\item {} 
Issue Comands using the \sphinxstylestrong{“echo”} command
echo ‘vipall?’ \textgreater{} /dev/ttyusb\#

\item {} 
Read the data with \sphinxstylestrong{cat}:
cat /dev/ttyusb\#

\end{itemize}


\section{First operations and tests}
\label{\detokenize{guide/getting_started:first-operations-and-tests}}
If the unit has been configured correctly


\section{First Troubleshouting}
\label{\detokenize{guide/getting_started:first-troubleshouting}}

\section{Notes and disclaimer}
\label{\detokenize{guide/getting_started:notes-and-disclaimer}}
If you find an error in this document, or have suggestions for how we could make it better, please do get in touch with us at \sphinxhref{mailto:support@qontrol.co.uk}{support@qontrol.co.uk} with your comments.

The information provided in this document is believed to be accurate at the time of publication. It is provided for information only, ‘as is’, and without guarantee of any kind.

Qontrol Systems LLP, its subsidiaries and associates accept no liability for damage to equipment, hardware, or the customer application, or for labour costs incurred due to the information contained in this document.


\chapter{qontrol\_api}
\label{\detokenize{modules:qontrol-api}}\label{\detokenize{modules::doc}}

\section{qontrol module}
\label{\detokenize{qontrol:module-qontrol}}\label{\detokenize{qontrol:qontrol-module}}\label{\detokenize{qontrol::doc}}\index{qontrol (module)@\spxentry{qontrol}\spxextra{module}}\index{ChannelVector (class in qontrol)@\spxentry{ChannelVector}\spxextra{class in qontrol}}

\begin{fulllineitems}
\phantomsection\label{\detokenize{qontrol:qontrol.ChannelVector}}\pysiglinewithargsret{\sphinxbfcode{\sphinxupquote{class }}\sphinxcode{\sphinxupquote{qontrol.}}\sphinxbfcode{\sphinxupquote{ChannelVector}}}{\emph{base\_list}}{}
Bases: \sphinxcode{\sphinxupquote{object}}

Custom list class which has a fixed length but mutable (typed) elements, and which phones home when its elements are read or modified.
\index{get\_handle (qontrol.ChannelVector attribute)@\spxentry{get\_handle}\spxextra{qontrol.ChannelVector attribute}}

\begin{fulllineitems}
\phantomsection\label{\detokenize{qontrol:qontrol.ChannelVector.get_handle}}\pysigline{\sphinxbfcode{\sphinxupquote{get\_handle}}\sphinxbfcode{\sphinxupquote{ = None}}}
\end{fulllineitems}

\index{set\_handle (qontrol.ChannelVector attribute)@\spxentry{set\_handle}\spxextra{qontrol.ChannelVector attribute}}

\begin{fulllineitems}
\phantomsection\label{\detokenize{qontrol:qontrol.ChannelVector.set_handle}}\pysigline{\sphinxbfcode{\sphinxupquote{set\_handle}}\sphinxbfcode{\sphinxupquote{ = None}}}
\end{fulllineitems}

\index{valid\_types (qontrol.ChannelVector attribute)@\spxentry{valid\_types}\spxextra{qontrol.ChannelVector attribute}}

\begin{fulllineitems}
\phantomsection\label{\detokenize{qontrol:qontrol.ChannelVector.valid_types}}\pysigline{\sphinxbfcode{\sphinxupquote{valid\_types}}\sphinxbfcode{\sphinxupquote{ = (\textless{}class \textquotesingle{}int\textquotesingle{}\textgreater{}, \textless{}class \textquotesingle{}float\textquotesingle{}\textgreater{})}}}
\end{fulllineitems}


\end{fulllineitems}

\index{QXOutput (class in qontrol)@\spxentry{QXOutput}\spxextra{class in qontrol}}

\begin{fulllineitems}
\phantomsection\label{\detokenize{qontrol:qontrol.QXOutput}}\pysiglinewithargsret{\sphinxbfcode{\sphinxupquote{class }}\sphinxcode{\sphinxupquote{qontrol.}}\sphinxbfcode{\sphinxupquote{QXOutput}}}{\emph{*args}, \emph{**kwargs}}{}
Bases: {\hyperref[\detokenize{qontrol:qontrol.Qontroller}]{\sphinxcrossref{\sphinxcode{\sphinxupquote{qontrol.Qontroller}}}}}
\index{get\_all\_values() (qontrol.QXOutput method)@\spxentry{get\_all\_values()}\spxextra{qontrol.QXOutput method}}

\begin{fulllineitems}
\phantomsection\label{\detokenize{qontrol:qontrol.QXOutput.get_all_values}}\pysiglinewithargsret{\sphinxbfcode{\sphinxupquote{get\_all\_values}}}{\emph{para=\textquotesingle{}V\textquotesingle{}}}{}
\end{fulllineitems}

\index{get\_value() (qontrol.QXOutput method)@\spxentry{get\_value()}\spxextra{qontrol.QXOutput method}}

\begin{fulllineitems}
\phantomsection\label{\detokenize{qontrol:qontrol.QXOutput.get_value}}\pysiglinewithargsret{\sphinxbfcode{\sphinxupquote{get\_value}}}{\emph{ch}, \emph{para=\textquotesingle{}V\textquotesingle{}}}{}
\end{fulllineitems}

\index{set\_all\_values() (qontrol.QXOutput method)@\spxentry{set\_all\_values()}\spxextra{qontrol.QXOutput method}}

\begin{fulllineitems}
\phantomsection\label{\detokenize{qontrol:qontrol.QXOutput.set_all_values}}\pysiglinewithargsret{\sphinxbfcode{\sphinxupquote{set\_all\_values}}}{\emph{para=\textquotesingle{}V\textquotesingle{}}, \emph{values=0}}{}
Convenience function for slicing up set commands into vectors for each module and transmitting.
\begin{quote}

para:          Parameter to set \{‘V’ or ‘I’\}
values:        Either float/int or list of float/int of length n\_chs
\end{quote}

\end{fulllineitems}

\index{set\_value() (qontrol.QXOutput method)@\spxentry{set\_value()}\spxextra{qontrol.QXOutput method}}

\begin{fulllineitems}
\phantomsection\label{\detokenize{qontrol:qontrol.QXOutput.set_value}}\pysiglinewithargsret{\sphinxbfcode{\sphinxupquote{set\_value}}}{\emph{ch}, \emph{para=\textquotesingle{}V\textquotesingle{}}, \emph{new=0}}{}
\end{fulllineitems}


\end{fulllineitems}

\index{Qontroller (class in qontrol)@\spxentry{Qontroller}\spxextra{class in qontrol}}

\begin{fulllineitems}
\phantomsection\label{\detokenize{qontrol:qontrol.Qontroller}}\pysiglinewithargsret{\sphinxbfcode{\sphinxupquote{class }}\sphinxcode{\sphinxupquote{qontrol.}}\sphinxbfcode{\sphinxupquote{Qontroller}}}{\emph{*args}, \emph{**kwargs}}{}
Bases: \sphinxcode{\sphinxupquote{object}}

Super class which handles serial communication, device identification, and logging.
\begin{quote}

device\_id = None                                        Device ID
serial\_port = None                                      Serial port object
serial\_port\_name = None                         Name of serial port, eg ‘COM1’ or ‘/dev/tty1’
error\_desc\_dict = Q8x\_ERRORS                    Error code descriptions
log = fifo(maxlen = 256)                        Log FIFO of sent commands and received errors
log\_handler = None                                      Function which catches log dictionaries
log\_to\_stdout = True                            Copy new log entries to stdout
response\_timeout = 0.050                        Timeout for response or error to commands
inter\_response\_timeout = 0.020          Timeout for response or error to get commands
\end{quote}

Log handler:
The log handler may be used to catch and dynamically handle certain errors, as they arise. In the following example, it is set up to raise a RuntimeError upon reception of errors E01, E02, and E03:
\begin{quote}

q = Qontroller()

fatal\_errors = {[}1, 2, 3{]}
\begin{description}
\item[{def my\_log\_handler(err\_dict):}] \leavevmode\begin{description}
\item[{if err\_dict{[}‘type’{]} is ‘err’ and err\_dict{[}‘id’{]} in fatal\_errors:}] \leavevmode
raise RuntimeError(‘Caught Qontrol error “\{1\}” at \{0\} ms’.format(1000*err\_dict{[}‘proctime’{]}, err\_dict{[}‘desc’{]}))

\end{description}

\end{description}

q.log\_handler = my\_log\_handler
\end{quote}
\index{close() (qontrol.Qontroller method)@\spxentry{close()}\spxextra{qontrol.Qontroller method}}

\begin{fulllineitems}
\phantomsection\label{\detokenize{qontrol:qontrol.Qontroller.close}}\pysiglinewithargsret{\sphinxbfcode{\sphinxupquote{close}}}{}{}
Release resources

\end{fulllineitems}

\index{issue\_binary\_command() (qontrol.Qontroller method)@\spxentry{issue\_binary\_command()}\spxextra{qontrol.Qontroller method}}

\begin{fulllineitems}
\phantomsection\label{\detokenize{qontrol:qontrol.Qontroller.issue_binary_command}}\pysiglinewithargsret{\sphinxbfcode{\sphinxupquote{issue\_binary\_command}}}{\emph{command\_id}, \emph{ch=None}, \emph{BCAST=0}, \emph{ALLCH=0}, \emph{ADDM=0}, \emph{RW=0}, \emph{ACT=0}, \emph{DEXT=0}, \emph{value\_int=0}, \emph{addr\_id\_num=0}, \emph{n\_lines\_requested=2147483648}, \emph{target\_errors=None}, \emph{output\_regex=\textquotesingle{}(.*)\textquotesingle{}}, \emph{special\_timeout=None}}{}
Transmit command ({[}command\_id{]}{[}ch{]}{[}operator{]}{[}value{]}) to device, collect response.
\begin{quote}

command\_id:             Command descriptor, either int (command index) or str (command name).
ch:                     Channel address (0x0000..0xFFFF for ADDM=0, 0x00..0xFF for ADDM=1).
BCAST,
\begin{quote}

ALLCH,
ADDM,
RW,
ACT,
DEXT:                  Header byte bits. See Programming Manual for full description.
\end{quote}

value\_int:              Data, either int (DEXT=0) or list of int (DEXT=1).
addr\_id\_num:    For device\sphinxhyphen{}wise addressing mode (ADDM=1) only, hex device ID code.
\end{quote}

All other arguments same as those for issue\_command()

\end{fulllineitems}

\index{issue\_command() (qontrol.Qontroller method)@\spxentry{issue\_command()}\spxextra{qontrol.Qontroller method}}

\begin{fulllineitems}
\phantomsection\label{\detokenize{qontrol:qontrol.Qontroller.issue_command}}\pysiglinewithargsret{\sphinxbfcode{\sphinxupquote{issue\_command}}}{\emph{command\_id}, \emph{ch=None}, \emph{operator=\textquotesingle{}\textquotesingle{}}, \emph{value=None}, \emph{n\_lines\_requested=2147483648}, \emph{target\_errors=None}, \emph{output\_regex=\textquotesingle{}(.*)\textquotesingle{}}, \emph{special\_timeout=None}}{}
Transmit command ({[}command\_id{]}{[}ch{]}{[}operator{]}{[}value{]}) to device, collect response.
\begin{quote}

command\_id                      Command header (e.g. ‘v’ in ‘v7=1.0’)
ch                                      Channel index to apply command to (e.g. ‘7’ in ‘v7=1.0’)
operator                        Type of command in \{?, =\} (e.g. ‘=’ in ‘v7=1.0’)
value                           Value of set command (e.g. ‘1.0’ in ‘v7=1.0’)
n\_lines\_requested       Lines of data (not error) to stop after receiving, or timeout
target\_errors           Error numbers which will be raised as RuntimeError
special\_timeout         Timeout to use for this command only (!= self.response\_timeout)
\end{quote}

\end{fulllineitems}

\index{log\_append() (qontrol.Qontroller method)@\spxentry{log\_append()}\spxextra{qontrol.Qontroller method}}

\begin{fulllineitems}
\phantomsection\label{\detokenize{qontrol:qontrol.Qontroller.log_append}}\pysiglinewithargsret{\sphinxbfcode{\sphinxupquote{log\_append}}}{\emph{type=\textquotesingle{}err\textquotesingle{}}, \emph{id=\textquotesingle{}\textquotesingle{}}, \emph{ch=0}, \emph{value=0}, \emph{desc=\textquotesingle{}\textquotesingle{}}, \emph{raw=\textquotesingle{}\textquotesingle{}}}{}
Append an event to the log, adding both a calendar\sphinxhyphen{} and a process\sphinxhyphen{}timestamp.”

\end{fulllineitems}

\index{parse\_error() (qontrol.Qontroller method)@\spxentry{parse\_error()}\spxextra{qontrol.Qontroller method}}

\begin{fulllineitems}
\phantomsection\label{\detokenize{qontrol:qontrol.Qontroller.parse_error}}\pysiglinewithargsret{\sphinxbfcode{\sphinxupquote{parse\_error}}}{\emph{error\_str}}{}
Parse an encoded error (e.g. E02:07) into its code, channel, and human\sphinxhyphen{}readable description.

\end{fulllineitems}

\index{print\_log() (qontrol.Qontroller method)@\spxentry{print\_log()}\spxextra{qontrol.Qontroller method}}

\begin{fulllineitems}
\phantomsection\label{\detokenize{qontrol:qontrol.Qontroller.print_log}}\pysiglinewithargsret{\sphinxbfcode{\sphinxupquote{print\_log}}}{\emph{n=None}}{}
Print the n last log entries. If n == None, print all log entries.

\end{fulllineitems}

\index{receive() (qontrol.Qontroller method)@\spxentry{receive()}\spxextra{qontrol.Qontroller method}}

\begin{fulllineitems}
\phantomsection\label{\detokenize{qontrol:qontrol.Qontroller.receive}}\pysiglinewithargsret{\sphinxbfcode{\sphinxupquote{receive}}}{}{}
Low\sphinxhyphen{}level receive data method which also checks for errors.

\end{fulllineitems}

\index{transmit() (qontrol.Qontroller method)@\spxentry{transmit()}\spxextra{qontrol.Qontroller method}}

\begin{fulllineitems}
\phantomsection\label{\detokenize{qontrol:qontrol.Qontroller.transmit}}\pysiglinewithargsret{\sphinxbfcode{\sphinxupquote{transmit}}}{\emph{command\_string}, \emph{binary\_mode=False}}{}
Low\sphinxhyphen{}level transmit data method. command\_string can be of type str or bytearray

\end{fulllineitems}

\index{wait() (qontrol.Qontroller method)@\spxentry{wait()}\spxextra{qontrol.Qontroller method}}

\begin{fulllineitems}
\phantomsection\label{\detokenize{qontrol:qontrol.Qontroller.wait}}\pysiglinewithargsret{\sphinxbfcode{\sphinxupquote{wait}}}{\emph{seconds=0.0}}{}
Do nothing while watching for errors on the serial bus.

\end{fulllineitems}


\end{fulllineitems}

\index{run\_interactive\_shell() (in module qontrol)@\spxentry{run\_interactive\_shell()}\spxextra{in module qontrol}}

\begin{fulllineitems}
\phantomsection\label{\detokenize{qontrol:qontrol.run_interactive_shell}}\pysiglinewithargsret{\sphinxcode{\sphinxupquote{qontrol.}}\sphinxbfcode{\sphinxupquote{run\_interactive\_shell}}}{}{}
Interactive shell for interacting directly with Qontrol hardware.

\end{fulllineitems}



\chapter{Indices and tables}
\label{\detokenize{index:indices-and-tables}}\begin{itemize}
\item {} 
\DUrole{xref,std,std-ref}{genindex}

\item {} 
\DUrole{xref,std,std-ref}{modindex}

\item {} 
\DUrole{xref,std,std-ref}{search}

\end{itemize}


\renewcommand{\indexname}{Python Module Index}
\begin{sphinxtheindex}
\let\bigletter\sphinxstyleindexlettergroup
\bigletter{q}
\item\relax\sphinxstyleindexentry{qontrol}\sphinxstyleindexpageref{qontrol:\detokenize{module-qontrol}}
\end{sphinxtheindex}

\renewcommand{\indexname}{Index}
\printindex
\end{document}